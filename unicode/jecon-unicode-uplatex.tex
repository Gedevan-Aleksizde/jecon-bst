%#!uplatex 
%#BIBTEX upbibtex jecon-unicode-uplatex.aux
% 
% このファイルは uplatex でコンパイルすることを前提で作成しています。
%
%############################## Main #################################

% uplatex でコンパイルするので、plautopatch パッケージを読み込む。
\RequirePackage{plautopatch}

%% jsarticle クラスを利用。
\documentclass[uplatex,dvipdfmx]{jsarticle}

\usepackage[T1]{fontenc}

\usepackage{natbib}
\usepackage{url}
\usepackage{amsmath,amssymb,amsthm}

%% Font を変更 → Times系に.
\usepackage{newtxtext,newtxmath}
%% 色を付ける.
\usepackage{graphicx}
\usepackage{color}
\definecolor{MyBrown}{rgb}{0.3,0,0}
\definecolor{MyBlue}{rgb}{0,0,0.3}
\definecolor{MyRed}{rgb}{0.6,0,0.1}
\definecolor{MyGreen}{rgb}{0,0.4,0}
\usepackage[dvipdfmx,bookmarks=true,%
bookmarksnumbered=true,%
colorlinks=true,%
linkcolor=MyBlue,%
citecolor=MyRed,%
filecolor=MyBlue,%
urlcolor=MyGreen%
]{hyperref}

% 以下の命令を入れておかないと \DH が定義されていないというエラーとなる。
% \renewcommand{\DH}{\fontencoding{T1}\selectfont{\symbol{240}}}

%#####################################################################
%######################### Document Starts ###########################
%#####################################################################
\begin{document}

\begin{flushleft}
 {\Large \textbf{upLaTeX (\texttt{uplatex}) を利用するケース}}
\end{flushleft}

\vspace{1em}

\section{使い方}

\subsection{TeX ファイルの書き方}

\begin{itemize}
 \item このファイル (\texttt{jecon-unicode-uplatex.tex}) を参考にしてください。
 \item \texttt{bib} ファイルの書き方、\texttt{jecon.bst} の使い方は
       \texttt{jecon-unicode-lualatex.pdf} を見てください。
\end{itemize}

\subsection{コンパイル}

\begin{itemize}
 \item upLaTeX でのコンパイル。
\begin{verbatim}
uplatex jecon-unicode-uplatex.tex
upbibtex jecon-unicode-uplatex
uplatex jecon-unicode-uplatex.tex               
uplatex jecon-unicode-uplatex.tex
dvipdfmx jecon-unicode-uplatex.dvi
\end{verbatim}
 \item つまり、\TeX のコマンドとしては \texttt{uplatex} を、BibTeX のコマンドと
       しては \texttt{upbibtex} を利用します。
 \item \texttt{uplatex} では DVI ファイルしか作成できませんので、最後に
       \texttt{dvipdfmx} で PDF に変換します。
\end{itemize}

\section{例}

\input{unicode-example.tex}

\nocite{*}

%%% BibTeX スタイルファイルの指定.jecon-unicode.bst を指定.
\bibliographystyle{jecon-unicode}

%% BibTeX データベースファイルの指定.
%
% jecon-example-reverse.bib は一個上のフォルダにあるものを利用する。
\bibliography{../jecon-example,unicode-example}

\end{document}
%#####################################################################
%######################### Document Ends #############################
%#####################################################################

% --------------------
% Local Variables:
% fill-column: 80
% coding: utf-8-dos
% End:

