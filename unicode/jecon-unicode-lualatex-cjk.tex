%#!lualatex
%#BIBTEX upbibtex jecon-unicode-lualatex-cjk
% 
% このファイルは lualatex でコンパイルすることを前提で作成しています。
% uplatex や xelatex でコンパイルするにはプリアンプルの設定等を変更する必要があ
% ります。
%
%############################## Main #################################

%% jlreq クラスを利用
\documentclass[article]{jlreq}

%% 以下は lualatex 用の設定。
\usepackage[no-math]{fontspec}
\usepackage{luatexja-fontspec}

% フォントに「源ノ角ゴシックCJK」と「源ノ明朝CJK」を利用
\usepackage[sourcehan]{luatexja-preset}

%% 以下は共通の設定
\usepackage[longnamesfirst]{natbib}
\usepackage{url}
\usepackage{amsmath,amssymb,amsthm}
\usepackage{ulem}

%% Font を変更 → Times系に.
\usepackage{newtxtext,newtxmath}
%% 色を付ける.
\usepackage{graphicx}
\usepackage{color}
\definecolor{MyBrown}{rgb}{0.3,0,0}
\definecolor{MyBlue}{rgb}{0,0,0.3}
\definecolor{MyRed}{rgb}{0.6,0,0.1}
\definecolor{MyGreen}{rgb}{0,0.4,0}
\definecolor{MyAltColor}{rgb}{0.6,0,0}
\usepackage[unicode=true,%
pdfusetitle=true,%
bookmarks=true,%
bookmarksnumbered=true,%
colorlinks=true,%
linkcolor=MyBlue,%
citecolor=MyRed,%
filecolor=MyBlue,%
urlcolor=MyGreen%
]{hyperref}

\makeatletter
%% \BibTeX command を定義.
\newcommand{\BibTeX}{\textrm{B\kern-.05em\textsc{i\kern-.025em b}\kern-.08em%
    T\kern-.1667em\lower.7ex\hbox{E}\kern-.125emX}}
\makeatother

%%% title, author, acknowledgement, and date
\title{\textbf{LuaLaTeX + 「源ノ角ゴシックCJK」+「源ノ明
 朝CJK」で日中韓の文字の混在}
\thanks{このファイルの配布場所: \url{https://github.com/ShiroTakeda/jecon-bst}}
}

\author{武田史郎\thanks{Email:
\texttt{\href{mailto:shiro.takeda@gmail.com}{shiro.takeda@gmail.com}}}}

\date{\today}

%#####################################################################
%######################### Document Starts ###########################
%#####################################################################
\begin{document}

% \maketitle

\begin{flushleft}
{\Large \textbf{LuaLaTeX + 「源ノ角ゴシックCJK」+「源ノ明
 朝CJK」で日中韓の文字の混在}} 
\end{flushleft}

\vspace*{1em}

このファイルでは「源ノ角ゴシックCJK」と「源ノ明朝CJK」のフォントの利用を前提とし
ています。
\begin{itemize}
 \item \url{https://github.com/adobe-fonts} の「source-han-serif」(こちらが明朝)
       と「source-han-sans」(こちらがゴシック)からダウンロードして、インストール
       できます。
\end{itemize}

\vspace*{1em}

bib ファイルにおける設定
\begin{itemize}
 \item 中国語、韓国語の文献は multi-lang.bib に登録されています。
 \item 中国語の文献には、\verb|language|フィールドに\verb|cn|か\verb|chn|を指定
       してください。\verb|language = {cn}| 、\verb|language = {chn}| のようにで
       す。
 \item 韓国語の文献には、\verb|language|フィールドに\verb|ko|か\verb|kor|を指定
       してください。\verb|language = {ko}| 、\verb|language = {kor}| のようにで
       す。
\end{itemize}

\section{引用例}
\begin{itemize}
 \item \citet{zhang_IO_2009}: 『教育-经济投入占用产出模型研究』
 \item \citet{40022221997}: 「关于用于句末的``期待''」
 \item \citet{40022198771}: 「浅析日本景点中文介绍中翻译的误区 : 以日本古城堡相关词汇为例」
 \item \citet{40022308104}: 「미시통계자료를 리용한 행동경제학의 실증분석과 재일조선인연구에로의 적용가능성」
 \item \citet{gwon-osang-env-econ-2020}: 『환경경제학 개정판 4판』
 \item \citet{Seong-Tae_Kim_2011}: 「한국의 산업별 생산의 대체탄력성 추정」
 \item \citet{arimura-takeda-energy-2017}: 『节能与排放量交易的经济分析:日本企业和家庭的现状』
 \item \citet{Stern_2016:chn}: 『尚待何时?应对气候变化的逻辑、紧迫性和前景』
 \item \citet{china-io-2007}: 『中国2007年投入产出表』
 \item \citet{Stokey2004}: ``Global Sourcing''
 \item \citet{Bohringer2006}: ``Computable General Equilibrium Models for
       Sustainability Impact Assessment: Status Quo and Prospects''
 \item \citet{内田90}: 『冥途・旅順入城式』
 \item \citet{120005614155}: 「学術研究のためのオープンソース・ソフトウェア (1)
       XELATEX(靎見誠良教授退職記念号)」
\end{itemize}

\section{問題点}

問題その1
\begin{itemize}
 \item 日本語の文献については、\texttt{yomi} フィールドによって著者名の読みを指
       定しています(ひらがなで)。これにより、日本人著者の文献は「五十音順」
       に並びます。
 \item しかし、中国語、韓国語の文献については著者名の読みを指定していないので、
       順序は「五十音順」にはならないです(たぶん、著者名の文字の文字コード順で
       す)。
\end{itemize}

問題その2
\begin{itemize}
 \item 中国語、韓国語の文献については、日本語文献と同じような見た目になるように
       しています。
 \item 姓名は「姓→名」の順。論文タイトルは「」、書籍のタイトルは『』で囲む。
 \item ですので、あくまで日本語の論文で中国語、韓国語の文献を引用するという前提
       です。
\end{itemize}

問題その3
\begin{itemize}
 \item それと韓国語については全く理解できないため、例として利用している文献もお
       かしくなっているかもしれません。適当にネットで検索して出てきた文献を使っ
       ています。名前の姓と名の区切も全くわからないので、ハングルの最初の一文字
       を姓として扱っています。
\end{itemize}



\vspace{1em}

% \section{例}

% \input{unicode-example.tex}

\nocite{*}

%%% BibTeX スタイルファイルの指定.jecon-unicode.bst を指定.
\bibliographystyle{jecon-unicode}

%% BibTeX データベースファイルの指定.
%
% jecon-example.bib は一個上のフォルダにあるものを利用する.
% \bibliography{../jecon-example,unicode-example,multi-lang}
\bibliography{unicode-example,multi-lang}

\end{document}
%#####################################################################
%######################### Document Ends #############################
%#####################################################################

% --------------------
% Local Variables:
% fill-column: 80
% coding: utf-8-dos
% End:

